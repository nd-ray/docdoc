\documentclass[sigchi-a, authorversion]{acmart}
\usepackage{booktabs} % For formal tables
\usepackage{ccicons}  % For Creative Commons citation icons
\usepackage{tabularx}

\newcommand{\comm}[1]{}

% Copyright
%\setcopyright{none}
\setcopyright{acmcopyright}
%\setcopyright{acmlicensed}
%\setcopyright{rightsretained}
%\setcopyright{usgov}
%\setcopyright{usgovmixed}
%\setcopyright{cagov}
%\setcopyright{cagovmixed}


% DOI
\acmDOI{10.475/123_4}

% ISBN
\acmISBN{123-4567-24-567/08/06}

%Conference
\acmConference[CHI'19 Extended Abstracts]{ACM CHI conference}{May 2019}{Glasgow, UK}
\acmYear{2019}
\copyrightyear{2016}

\acmPrice{15.00}

%\acmBadgeL[http://ctuning.org/ae/ppopp2016.html]{ae-logo}
%\acmBadgeR[http://ctuning.org/ae/ppopp2016.html]{ae-logo}

\begin{document}
\title{Outstanding: A Perspective- Switching Technique for Covering Large Distances in VR Games}

\author{Sebastian Cmentowski}
\affiliation{%
\institution{High Performance Computing Group}
  \institution{University of Duisburg-Essen, Germany}
}
\email{sebastian.cmentowski@uni-due.de}

\author{Andrey Krekhov}
\affiliation{%
\institution{High Performance Computing Group}
  \institution{University of Duisburg-Essen, Germany}
}
\email{andrey.krekhov@uni-due.de}

\author{Jens Kr\"uger}
\affiliation{%
\institution{High Performance Computing Group}
  \institution{University of Duisburg-Essen, Germany}
}
\email{jens.krueger@uni-due.de}



% The default list of authors is too long for headers.
\renewcommand{\shortauthors}{S. Cmentowski et al.}
\settopmatter{printacmref=false}


\begin{marginfigure}
     \begin{minipage}{\marginparwidth} 
    \includegraphics[width=\marginparwidth]{OutsideAvatar/images/110_Travel1.jpg}
    \vspace{0.25em}
    
    \includegraphics[width=\marginparwidth]{OutsideAvatar/images/110_Travel2.jpg}
    \vspace{0.25em}
    
    \includegraphics[width=\marginparwidth]{OutsideAvatar/images/110_Travel3.jpg}
    \end{minipage}
    \caption{Navigation Technique: Start in \textit{first-person} (1), grow to \textit{third-person perspective} and set a navigation target (2), wait for the avatar to walk there and switch back to \textit{first-person} (3).}~\label{fig:nvigation}
\end{marginfigure}


\begin{abstract}

Room-scale virtual reality games allow players to experience an unmatched level of presence. A major reason is the natural navigation provided by physical walking. However, the tracking space is still limited, and viable alternatives or extensions are required to reach further virtual destinations. Our current work focuses on traveling over (very) large distances--an area where approaches such as teleportation are too exhausting and WIM teleportations potentially reduce presence. Our idea is to equip players with the ability to switch from first-person to a third-person god-mode perspective on demand. From above, players can command their avatar similar to a real-time strategy game and initiate travels over large distance. In our first exploratory evaluation, we learned that the proposed dynamic switching is intuitive, increases spatial orientation, and allows players to maintain a high degree of presence throughout the game. Based on the outcomes of a  participatory design workshop, we also propose a set of extensions to our technique that should be considered in the future.





\end{abstract}


  
\keywords{Virtual reality games; navigation; fast-travel; virtual avatar; presence; world-in-miniature}
\maketitle

\begin{sidebar}
\textit{"I did not even have to think about navigating, everything was completely natural."~(\textbf{P1})}

\vspace{2.5em}
\begin{flushright}\textit{"I could see so much more during traveling - this invited me to explore interesting spots."~(\textbf{P2})}\end{flushright}

\vspace{2.5em}
\textit{"I liked that the travel speed was limited by the avatar's walking pace; it made everything very realistic."~(\textbf{P3})}

\vspace{2.5em}
\begin{flushright}\textit{"When I realized that I was able to leave the path and explore the world freely on my own: that was the moment the game became so interesting to experience."~(\textbf{P4})}\end{flushright}

\vspace{2.5em}
\textit{"The avatar intelligently chose to take shortcuts, this is an awesome technique!"~(\textbf{P5})}

\vspace{2.5em}
\begin{flushright}\textit{"This felt like being god and commanding entities, with the ability to possess them at will."~(\textbf{P6})}\end{flushright}

\vspace{2.5em}
\textit{"The technique enabled me to travel large distances with one click - this was so easy and comfortable to use!"~(\textbf{P7})}

\vspace{2.5em}
\begin{flushright}\textit{"I always knew where I was and which path to take. The overview that I gained through traveling was excellent!"~(\textbf{P8})}\end{flushright}
\end{sidebar}

\section{Introduction}

Virtual reality allows players to explore fictive environments in an immersive and natural manner, to experience a feeling of being there, and to almost forget the real surrounding. But what does it take to realize virtual experiences of the vast and open worlds usually found in today's digital games? With continuous technical improvements such as higher display resolutions, better spatial tracking, and new rendering techniques, current game engines are capable of rendering even larger scenes despite the necessary VR overhead. However, stable frame rates and huge open environments are useless without proper techniques that enable players to freely and immersively wander through these landscapes. Natural walking using room-scale tracking is too confined by the available space. Typically used virtual locomotion techniques are either optimized for short distances, prone to cybersickness or involve visible cuts that decrease the perceived presence. To our knowledge, there is currently no available navigation approach that was specifically designed for large-distance travels and that preserves high levels of presence and spatial orientation while avoiding cybersickness.\par
We propose a navigation technique that fills this gap by using multiple player perspectives. Players can dynamically switch between a first-person and third-person god-mode perspective depending on the current task: If players want to explore a local spot and interact with the environment, they can use a first-person point-of-view to experience the world like they are used to. In the third-person mode, they watch and command their avatar from a bird's eye perspective, which allows them to cover larger distances with ease. We claim that this technique does not infer any cybersickness and increases the perceived presence compared to established techniques.\par In this paper, we report our ongoing work on the previously described navigation approach. After realizing an implementation of the technique and a suitable test bed scenario, we have gone through an extensive participatory design phase to optimize and tweak the available parameters. The resulting prototype was evaluated in an exploratory study to gain further insights into usage patterns, problems, and necessary additions. \comm{Backed by our evaluation, we propose a number of application possibilities and discuss the related benefits and drawbacks.}

\section{Related Work}
Most non-VR games use joysticks\comm{ or keyboards} to control\comm{ the movement of} the player's avatar. Such approaches involving continuous\comm{ virtual} motion are rarely transferable to VR as they tend to induce cybersickness~\cite{Habgood:2017:HLP:3130859.3131437}. Instead, many VR games rely on natural walking~\cite{ruddle2009benefits} to achieve natural and presence-preserving navigation. However, the confined \comm{supported }space of currently available room-scale tracking limits natural walking to few square meters. \par
Recent research has attempted to overcome this limitation by extending the range of real walking to enable the player to reach further. \citet{ Bhandari:2017:LSW:3139131.3139133} combined walking with walking in place and reported higher presence compared to traditional controller input.
\begin{marginfigure}
     \includegraphics[width=\marginparwidth]{OutsideAvatar/images/102_GlobalMap2.jpg}
     \caption{The medieval scenario used as testbed game. Players could follow the long path (marked in green) and search for animals at points of interest (bottom).}~\label{fig:world}
\end{marginfigure}
\begin{marginfigure}
     \includegraphics[width=\marginparwidth]{OutsideAvatar/images/103_Animals2.jpg}
     \vspace{-0.5em}
     
     \includegraphics[width=\marginparwidth]{OutsideAvatar/images/103_Animals1.jpg}
     \caption{Some of the animals found along the path. \textit{Forced Switching} is used to prevent players from missing these spots.}~\label{fig:animals}
\end{marginfigure}
Another approach by~\citet{bolte2011jumper} uses the detection of physical jumping: When an acceleration and consecutive jump are detected, the resulting forward motion is augmented to travel larger distances. \comm{Similarly,~\citet{ interrante2007seven} proposed their Seven League Boots metaphor that deduces and augments the intended travel direction while leaving all other directions unchanged.}\par
In contrast to these augmented walking approaches, purely virtual navigation techniques sacrifice the advantages of natural walking to achieve unlimited traveling. The most prominent approach is the teleportation technique: players aim at an accessible destination and are directly teleported there. This approach has been shown to be superior to the traditional gamepad locomotion~\cite{frommel2017effects}, however, the perceived presence and spatial orientation are significantly lowered by the instant relocation. Even worse, the necessity to view the target location limits the maximal distance to be traversed in one jump and vastly increases the necessary workload for larger travels or occluded areas. \par
One remedy for true large distance travel in VR is the concept of a world-in-miniature (WIM)~\cite{stoakley1995virtual}: a virtual three-dimensional minimap is \comm{shown on the player's hand and can be }used to move instantaneously to any point within a large and complex environment. \comm{This concept has been further refined by~\citet{laviola2001hands} to achieve a walkable minimap that is grown around the players feet to replace the previous environment.} In comparison to teleport, WIM works better with larger distances and occlusions~\cite{berger2018wim}. However, it introduces the minimap as an additional artificial interface which is decoupled from the original virtual world. Since both approaches were never designed to be a perfect solution for long-distance travel, this encouraged us to develop an immersive and natural alternative.\par
Our approach is based on switching between first-person (1PP) and third-person (3PP) perspectives. \comm{After early studies on the potential use of 3PP in virtual environments~\cite{salamin2006benefits},} \citet{gorisse2017first} administered the perceptual differences\comm{ between both views in an extensive study}: According to their experiments, both perspectives are able to preserve high levels of presence and agency. However, 1PP is best suited for interaction-intensive tasks and scenarios involving body ownership. In contrast, 3PP provides advantages to spatial awareness and environment perception. These findings support our idea for a navigation metaphor using 3PP for large-distance navigation and 1PP for local interaction. \citet{gorisse2017first} decided to place the 3PP viewpoint directly behind the avatar. However, this is not suitable in our case as it does not improve the view distance or environmental knowledge of the user. Instead, we have decided to scale the disembodied players to giant size. By doing so, the virtual camera position is moved to a greater height and enables players to perceive the environment as a miniature world. Meanwhile, their avatar resides at his original size to the feet of the players. This approach has been shown to be uncritical regarding cybersickness and provide benefits for spatial orientation~\cite{krekhov2018gullivr}. 

\section{Navigation Technique}
The main idea behind our locomotion technique is to switch different perspectives on demand based on the current situation. The first-person view of the normal mode (NM) is used for basic interaction and short-range exploration by physical walking within the virtual world. The travel mode (TM) is a third-person perspective for long-distance travels where players are scaled to ten times their original size and see their own avatar keeping his original scale and symbolizing the players first-person position in the world. The disembodied players can command the avatar by setting navigation targets through raycast aiming (cf. Figure~\ref{fig:nvigation}).
\begin{marginfigure}
     \begin{minipage}{\marginparwidth} 
    \includegraphics[width=\marginparwidth]{OutsideAvatar/images/104_Parameters.jpg}
    \vspace{-1em}
    \begin{flushleft}\caption{Transition Parameters.}~\label{fig:transitionParameters}\end{flushleft}
    \vspace{-1em}
    \textbf{Scale:} The factor by which players are scaled in travel mode. (cf. Figure~\ref{fig:scales}) \textit{Proposed value: 10x}
    
    \textbf{Offset:} The offset that is applied to the players' position in TM to improve visibility of the avatar and prevent a $90^\circ$ angle downwards. \textit{Proposed value: 20m which equals roughly $45^\circ$ }
    
    \textbf{Speed:} The total duration of the transition.   \textit{Proposed value: 0.5s}
    
    \textbf{Curve:} The applied animation curve used to transform between NM and TM. We tested linear and curved transitions.
    \vspace{1em}
    \end{minipage}
\end{marginfigure}
\begin{marginfigure}
    \includegraphics[width=\marginparwidth]{OutsideAvatar/images/105_Scalings.jpg}
    \caption{Different scaling factors for TM that were evaluated in the participatory design phase: $10x$, $30x$, and $100x$. Players preferred $10x$, even though this requires fixing the occlusion, e.g., of leaves.}~\label{fig:scales}
\end{marginfigure}
They can decide to switch back to NM at any time to explore a specific place in greater detail. \comm{The resulting transition between NM and TM is a short transformation between both camera position.} This concept uses both perspectives~\cite{gorisse2017first} to combine the fast and easy traveling of long distances with the possibility to explore details on demand. Due to the absence of instant relocations and artificial camera movements, we argue that this technique - in contrast to teleport and WIM - increases presence and prevents cybersickness. As positive side effect, we assume a significant gain in spatial orientation, as players are able to observe the surrounding world using a god mode and can choose an optimal path depending on this additional information.\par
Our proposed technique can be split into three components: NM, TM, and the transition between both states. While NM and TM only differ in the positioning of the virtual camera, the transition is more complex. It requires an optimal calibration to avoid cybersickness while preserving a consistent experience without noticeable cuts. The resulting animation depends on several continuous parameters (cf. Figure~\ref{fig:transitionParameters}) that need to be tweaked accordingly to achieve the desired outcome. 

\subsection{Participatory Design}
We executed an early participatory design phase to optimize the available continuous parameters. The 4 participants (2 female) were familiar with VR systems but not involved in this project. We asked them to test our provided scenario and give oral feedback which was used to calibrate the different parameters. We followed the order given in Figure~\ref{fig:transitionParameters}, starting from the most noticeable factor, and waited at least two minutes between two parameters. The result was a set of optimized parameters (cf. Figure~\ref{fig:transitionParameters}). At this stage, we also added two additional extensions: \textit{Forced Switching} and \textit{Catching Up}. \textit{Forced Switching} prevents players from missing out important destinations by transitioning them into NM (cf. Figure~\ref{fig:forcedSwitching}), while \textit{Catching Up}, triggered by a button press, moves the player`s camera to the current avatar position to reduce the necessary switches during longer travels (cf. Figure~\ref{fig:catchingUp}).

\subsection{Exploratory Study}
After the design phase, we executed an exploratory study to gather insights into how players would use our technique. This included preferences, expectations, and problems. The provided scenario was set in a fictive medieval world (cf. Figure~\ref{fig:world}) where players had to follow a long, twisted path passing several forests, lakes, villages, and mountains. The task was to find and observe different types of animals roaming specific areas (cf. Figure~\ref{fig:animals}). This setting provided a suitable testbed for our navigation metaphor as it included traveling two kilometers while switching back to NM at points of interest.\par


The study was conducted in our VR lab and took 45 minutes on average. \comm{After assessing the general familiarity with VR games, we introduced the subjects to our setup and started the game.} During the first half of the virtual travel, we encouraged the participants to give feedback. In the second part, we avoided any further conversations to allow the subjects to play undisturbed for a longer time and immerse themselves into the virtual environment. After finishing the game, the subjects had to fill out the Presence Questionnaire (PQ)~\cite{UQO.2004} and Simulator Sickness Questionnaire (SSQ)~\cite{kennedy1993simulator} to assess presence, cybersickness, and fun. The study was completed by another round of interview questions and the opportunity to share feedback, concerns or ideas regarding our approach.

\subsection{Results and Discussion}

\begin{marginfigure}
    \includegraphics[width=\marginparwidth]{OutsideAvatar/images/106_CatchingUp.jpg}
    \caption{With \textit{Catching Up}, players can close the large gap to their avatar (left) with a single button click (right).}~\label{fig:catchingUp}
\end{marginfigure}

\begin{marginfigure}
    \includegraphics[width=\marginparwidth]{OutsideAvatar/images/107_ForcedSwitching.jpg}
    \caption{If the avatar reaches an important destination, \textit{Forced Switching} is used to switch back to NM automatically.}~\label{fig:forcedSwitching}
\end{marginfigure}

In total, 8 persons (4 female) participated in our study with a mean age of 26.7 ($SD=7.98$). All participants reported playing digital games at least a few times a month and already had used VR gaming systems before. Even though this group size is too small for a quantitative analysis, the assessed questionnaires and given oral feedback already provide important insights.\par
The technique was very well accepted by all players and described as \textit{"natural" (\textbf{P4})} and \textit{"surprisingly intuitive"(\textbf{P5})}. Players were able to explore the world freely and could \textit{"reach every desired destination effortlessly"(\textbf{P2})}. The environment was generally reported as being fun to experience and interesting to explore. This feedback reflects the high scores of all subscales from PQ (cf. Figure~\ref{tab:PQ}). Additionally, the results from the SSQ (cf. Table~\ref{tab:SSQ}) suggest that using our technique causes no cybersickness.\par
Interestingly, the participants did not perceive the third-person avatar as their own body. Instead, they described it as an \textit{"NPC entity"(\textbf{P6})} controlled from a \textit{"god-mode"(\textbf{P5})}. However, this lack of body ownership was by no means a drawback but an opportunity for additional features such as controlling multiple entities or interacting with the world using god-powers. Additionally, players liked being constrained by the avatar as it made traveling \textit{"a realistic and believable process"} instead of an \textit{"artificial relocation"(\textbf{P3})}. However, all subjects initially expected their avatar to keep on walking after switching back from TM to NM: in our implementation, the avatar is stopped immediately after switching since artificial motion tends to cause severe cybersickness.\par
The optimized parameters from the participatory design phase were approved by all participants. The chosen scaling factor \textit{"balanced the spatial overview and the visibility of details"(\textbf{P1})} and invited players to \textit{"explore the world"} as it revealed \textit{"interesting spots in a greater context"(\textbf{P2})}. Even though the navigation worked best for larger distances, it \textit{"should be completed by other techniques like the teleport for local hotspots"(\textbf{P2})}. The only drawback of the chosen values was that the players had the same virtual height as the forest trees and ended up being surrounded by leaves. The resulting occlusion was perceived as disturbing. Possible solutions suggested by various players were to increase the scaling factor in forests, cull nearby leaves, or add a focus technology to reveal the avatar's position.\comm{ Interestingly, two participants preferred to keep the status quo as it forced them to actively evade trees and leaves by real walking in TM which felt \textit{"realistic"(\textbf{P1})}.}\par
Finally, we asked the subjects for potential VR games that could profit from such controls. The answers were plentiful and ranged from exploratory adventure games to simulations like \textit{The Sims} or role-playing games such as \textit{The Witcher} or \textit{Kingdom Come: Deliverance}. In general, the participants claimed a good benefit for explorations and huge open worlds but mentioned potential issues when trying to include real-time first-person actions like fights or dialogues.


\section{Conclusion and Future Work}

\begin{marginfigure}
\vspace{-4em}
    \includegraphics[width=\marginparwidth]{OutsideAvatar/images/109_Occlusion.jpg}
    \caption{Potential fix for the avatar occlusion through leaf-culling.}~\label{fig:occlusion}
\end{marginfigure}

\begin{marginfigure}
\includegraphics[width=\marginparwidth]{OutsideAvatar/images/D1_PQ.png}
\vspace{-2em}
\caption{Mean scores and standard deviations of the Presence Questionnaire (PQ). Subscale scores range from 0 (fully disagree) to 6 (fully agree).}
  \label{tab:PQ}
\vspace{0.5em}
\end{marginfigure}


%Table SSQ Values
\begin{margintable}
  
  \begin{tabularx}{\marginparwidth}{lXc}
    \toprule
    SSQ Dimension & & M (SD)\\
    \midrule
    \ \ Nausea & & 2.39 (4.77)\\
    \ \ Oculomotor & & 7.58 (15.16)\\
    \ \ Disorientation & & 10.44 (20.88)\\
    \ \ Total & & 6.80 (11.58)\\
  \bottomrule
\end{tabularx}
\vspace{0.5em}
\caption{Mean scores and standard deviations of the Simulator Sickness Questionnaire (SSQ).}
  \label{tab:SSQ}
\end{margintable}


We proposed a novel approach to large-distance travel in virtual environments, which uses the benefits of multiple perspectives to achieve natural and intuitive locomotion. In contrast to existing methods such as teleportation, players do not have to beam themselves hundreds of times but are able to traverse long distances with ease. Our exploratory study supports the basic assumption that this experience comes with high levels of presence, fun, and spatial orientation while completely avoiding cybersickness. As future steps, we suggest add requested features that emerged during the study. This includes a solution to the avatar occlusion (cf. Figure~\ref{fig:occlusion}), an option to walk faster, and the use of additional navigation methods for local exploration. Our next major step is to conduct a quantitative study to compare the proposed technique against existing approaches, to point out the relevant benefits and drawbacks, and to generate a comprehensive set of design guidelines to be used by researchers and practitioners.



\bibliography{Literatur}
\bibliographystyle{ACM-Reference-Format}

\end{document}
