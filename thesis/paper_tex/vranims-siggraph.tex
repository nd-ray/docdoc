\documentclass[acmtog,anonymous,review]{acmart}
\acmSubmissionID{232}

\usepackage{booktabs} % For formal tables

% TOG prefers author-name bib system with square brackets
\citestyle{acmauthoryear}
%\setcitestyle{nosort,square} % nosort to allow for manual chronological ordering



\usepackage[ruled]{algorithm2e} % For algorithms
\renewcommand{\algorithmcfname}{ALGORITHM}
\SetAlFnt{\small}
\SetAlCapFnt{\small}
\SetAlCapNameFnt{\small}
\SetAlCapHSkip{0pt}

% Metadata Information
\acmJournal{TOG}
%\acmVolume{38}
%\acmNumber{4}
%\acmArticle{39}
%\acmYear{2019}
%\acmMonth{7}

% Copyright
%\setcopyright{acmcopyright}
%\setcopyright{acmlicensed}
%\setcopyright{rightsretained}
%\setcopyright{usgov}
%\setcopyright{usgovmixed}
%\setcopyright{cagov}
%\setcopyright{cagovmixed}

% DOI
%\acmDOI{0000001.0000001_2}

% Paper history
%\received{February 2007}
%\received{March 2009}
%\received[final version]{June 2009}
%\received[accepted]{July 2009}

\include{ivda-macros}

% Document starts
\begin{document}
% Title portion
\title{VR Animals: The Illusion of Animal Body Ownership and Its Potential for Virtual Reality Applications}



% DO NOT ENTER AUTHOR INFORMATION FOR ANONYMOUS TECHNICAL PAPER SUBMISSIONS TO SIGGRAPH 2019!
%\author{Gang Zhou}
%\orcid{1234-5678-9012-3456}
%\affiliation{%
%  \institution{College of William and Mary}
%  \streetaddress{104 Jamestown Rd}
%  \city{Williamsburg}
%  \state{VA}
%  \postcode{23185}
%  \country{USA}}
%\email{gang_zhou@wm.edu}
%\author{Valerie B\'eranger}
%\affiliation{%
%  \institution{Inria Paris-Rocquencourt}
%  \city{Rocquencourt}
%  \country{France}
%}
%\email{beranger@inria.fr}
%\author{Aparna Patel}
%\affiliation{%
% \institution{Rajiv Gandhi University}
% \streetaddress{Rono-Hills}
% \city{Doimukh}
% \state{Arunachal Pradesh}
% \country{India}}
%\email{aprna_patel@rguhs.ac.in}
%\author{Huifen Chan}
%\affiliation{%
%  \institution{Tsinghua University}
%  \streetaddress{30 Shuangqing Rd}
%  \city{Haidian Qu}
%  \state{Beijing Shi}
%  \country{China}
%}
%\email{chan0345@tsinghua.edu.cn}
%\author{Ting Yan}
%\affiliation{%
%  \institution{Eaton Innovation Center}
%  \city{Prague}
%  \country{Czech Republic}}
%\email{yanting02@gmail.com}
%\author{Tian He}
%\affiliation{%
%  \institution{University of Virginia}
%  \department{School of Engineering}
%  \city{Charlottesville}
%  \state{VA}
%  \postcode{22903}
%  \country{USA}
%}
%\affiliation{%
%  \institution{University of Minnesota}
%  \country{USA}}
%\email{tinghe@uva.edu}
%\author{Chengdu Huang}
%\author{John A. Stankovic}
%\author{Tarek F. Abdelzaher}
%\affiliation{%
%  \institution{University of Virginia}
%  \department{School of Engineering}
%  \city{Charlottesville}
%  \state{VA}
%  \postcode{22903}
%  \country{USA}
%}

%\renewcommand\shortauthors{Zhou, G. et al}

\begin{abstract}
Virtual reality offers the unique possibility to experience a virtual representation as our own body. In contrast to previous research that predominantly studied this phenomenon for humanoid avatars, our work focuses on virtual animals. In this paper, we discuss different body tracking approaches to control creatures such as spiders or bats and the respective virtual body ownership effects. Our empirical results demonstrate that virtual body ownership is also applicable for nonhumanoids and can even outperform human-like avatars in certain cases. Two additional surveys confirm the general interest of people in creating such experiences and allow us to initiate a broad discussion regarding the applicability of animal embodiment. In particular, we outline the potential of such research for both educational and entertainment purposes and as a conceivable add-on in treatment of animal phobias.
\end{abstract}


%
% The code below should be generated by the tool at
% http://dl.acm.org/ccs.cfm
% Please copy and paste the code instead of the example below.
%
\begin{CCSXML}
<ccs2012>
<concept>
<concept_id>10003120.10003121.10003124.10010866</concept_id>
<concept_desc>Human-centered computing~Virtual reality</concept_desc>
<concept_significance>500</concept_significance>
</concept>
<concept>
<concept_id>10011007.10010940.10010941.10010969.10010970</concept_id>
<concept_desc>Software and its engineering~Interactive games</concept_desc>
<concept_significance>500</concept_significance>
</concept>
<concept>
<concept_id>10011007.10010940.10010941.10010969</concept_id>
<concept_desc>Software and its engineering~Virtual worlds software</concept_desc>
<concept_significance>100</concept_significance>
</concept>
</ccs2012>
\end{CCSXML}

\ccsdesc[500]{Human-centered computing~Virtual reality}
\ccsdesc[500]{Software and its engineering~Interactive games}
\ccsdesc[100]{Software and its engineering~Virtual worlds software}


\keywords{Virtual creatures, animal avatar control, virtual body ownership, animal embodiment, virtual reality games}

\begin{teaserfigure}
  \includegraphics[width=\textwidth]{figures/teaser}
  \caption{Although having great potential for education, entertainment, or phobia treatments, animal avatars are still a rarity in VR. We outline how such avatars can be controlled, explore the Illusion of Virtual Body Ownership for non-humanoids, and discuss related application areas.}
  \label{fig:teaser}
\end{teaserfigure}

\maketitle

\section{Introduction}




Due to the depth of immersion, VR setups often excel at creating a strong bond between users and their virtual representations, the so-called avatars. That bond can be strong enough such that we start perceiving the avatar model as our own body---a phenomenon also known as the illusion of virtual body ownership (IVBO)~\cite{slater2010first}. Previous research agrees that VR is an efficient setup to induce IVBO experiences~\cite{slater2009inducing,slater2010first,waltemate2018impact}. However, the investigated scenarios have been centered mostly around humanoid avatars. Our paper aims at generalizing the IVBO discussion by considering virtual animals as candidates for an embodiment experience.

To provide a starting ground for future research regarding animal embodiment in VR, our work addresses the following question: \textit{Is IVBO applicable to nonhumanoid avatars, and, if so, what potential does that phenomenon have for VR applications?} 

The primary contribution of our paper is the dedicated research of animal body ownership. Although prior work examined aspects such as the inclusion of additional limbs, tails, or wings, the question whether and how well we can embody virtual creatures remained unanswered. Our work provides a strong evidence that animal avatars can keep up and even outperform humanoid representations regarding IVBO. In our evaluation ($N=26$), we included a diversified set of animals to account for upright/flying species (bat), four-legged mammals (tiger), and arthropods (spider). Our experiment shows that even spiders, despite having a skeleton that significantly differs from ours, offer a similar degree of IVBO compared to humanoid avatars. Apart from the general assessment of IVBO, our paper proposes and discusses practical approaches (cf. \FG{fig:teaser}) to implement animal avatar control by, e.g., half-body tracking for non-upright avatars to reduce fatigue from crouching. We believe that our findings pave the way to the construction of a zoological IVBO framework in the future.


%Consequently, the first part---and the first contribution---of our paper is centered around an empirical study of the IVBO phenomenon applied to nonhumanoids. In particular, we propose five approaches to control different animals such as tigers, spiders, and bats (cf. \FG{fig:teaser}). Our evaluation ($N=26$) has promising outcomes, as the IVBO effect for animals keeps up with a human avatar and even outperforms the latter in certain cases.

Our additional contribution is a discussion about the potential of VR animals. We conducted an online survey ($N=37$) that underpins the general interest of people in experiencing virtual animals---be it in educational documentaries, for increasing our empathy regarding animal suffering, or as protagonists in VR games. An additional survey ($N=21$) targeted subjects with fear of spiders. Although we cannot forecast the effectivity of a IVBO-based treatment method, our subjects expressed less fear and more interest in trying such an approach compared to traditional VR treatment techniques.
 
Those two surveys support our claim that VR has a potential regarding animal embodiment, resulting in application possibilities for a number of core CG and HCI areas. Combined with our empirical findings that IVBO is applicable for nonhumanoids, our work paves the way for further, domain-specific research on virtual animals.

%Compared to traditional VR treatment approaches, potential candidates exposed less 
%
%
%
% Our insights are based on two additional surveys and cover the areas of education, entertainment, and treatments of animal phobias.
%
% The surveys underpin the general interest of people in experiencing virtual animals---be it in educational documentaries, for increasing our empathy regarding animal suffering, or as protagonists in VR games.
%
%
%
%Therefore, we conducted two additional surveys aiming at capturing 
%
%based on two additional surveys that explore the interest of people in experiencing virtual animals.
%
%%We consider full body and half body tracking modes as well as first and third person perspectives.
%
%First, we examine whether IVBO is applicable to non-humanoid avatars. Second, we discuss the overall potential and applicability of such a phenomenon regarding virtual reality research and applications.
%
% 
%
%However, past IVBO research mainly focused on humanoid avatars, as
%
%Hence, our ability to identify with an avatar plays an important role during the development of a VR application, and related decisions can potentially enhance or downgrade or experience.
%
%
%
%as the related design decisions 
%
%That phenomenon
%
%Virtual reality allows us to develop 
%
%Avatars play an important role in virtual reality setups
%
%intro: tell that vbop is cool but not well researched for animals. Q: why is animal vbo interesting? Education, games and entertainment, treatment of phobias. Contrib: evaluations confirm that. 
%
%contrib: answer two questions: is there animal vbo? And what can we do with it?
%
%
%\textbf{Old version:}
%
%
%Who do we want to be in a game? Sorcerer, rogue, or warrior, these are default roles most gamers would think of. Even when a game offers more exotic choices for our avatar, we usually still get a humanoid representation. Some successful games allow players to at least partly identify with realistic, non-humanoid avatars. For example, \textit{Black \& White}~\cite{BlackWhite} incorporated bipedal animals, \textit{Gothic 2}~\cite{Gothic} allowed players to transform into wild animals like wolves and scavengers to prevent fights or complete quests, and \textit{Deadly Creatures}~\cite{Deadly} offered combat experiences vs. other animals while being a tarantula or a scorpion. And although players seemed to enjoy such non-humanoid mechanics, playable, realistic animals remain a rarity in common digital games, not to mention VR titles. 
%
%
%Our claim is that incorporating animals as player avatars into VR has the potential to unveil a set of novel game mechanics and maybe even lead to a ``beastly'' VR game genre. Our claim is based on two assumptions. First, VR exposes the phenomenon of virtual body ownership (IVBO)~\cite{slater2010first}. We suppose that IVBO is transferable to non-humanoid characters, e.g., it is possible to experience the feeling of being in an animal body while playing a VR game. We suggest that such a feeling might be an interesting player experience worth further study. Second, utilizing the abilities of animals such as flying as a bird or crawling as a spider could be significantly more engaging in VR due to the increased presence compared to non-VR games.
%
%Hence, our first step into this direction was to explore whether and how IVBO applies to animals in VR. We also researched how players could control their representation in such games, as there often is no straightforward mapping between human and animal postures. Therefore, we have implemented two first person and three third person control modes for three different animals. We evaluated our mechanisms in an exploratory survey with eight participants. In addition, we repeated parts of the study with two children, as we were curious regarding their IVBO experiences and potential differences compared to adults. Our results indicate that IVBO is indeed noticeable even with non-humanoid models, and the interview outcomes expose the general interest of participants regarding playing as animals with the respective set of virtual skills.
%
%The contribution of our work in progress report is twofold. Backed by our evaluation, we propose several possibilities how animal avatars could be controlled in VR and what advantages and disadvantages such modes could have. Furthermore, our paper describes a number of possible future experiments that we think are helpful to better understand the IVBO phenomenon in VR games and to establish animal control as an engaging game mechanic.

\section{Application Areas}
\label{section:application}

To motivate the presented research, we briefly discuss how our community could benefit from animal avatars in VR and what possible research directions could be considered for follow-up explorations.

\subsection{Entertainment and Education}

Who do we want to be in a game? Sorcerer, rogue, or warrior--these are default roles most gamers would think of. Even when a game offers more exotic choices for our avatar, we usually still get a humanoid representation. Some successful games allow players to at least partly identify with realistic, nonhumanoid avatars. For example, \textit{Black \& White}~\cite{BlackWhite} incorporates bipedal animals, \textit{Gothic 2}~\cite{Gothic} allows players to transform into wild animals like wolves and scavengers to prevent fights or complete quests, and \textit{Deadly Creatures}~\cite{Deadly} offers combat experiences vs. other animals while being a tarantula or a scorpion. And although players seemed to enjoy such nonhumanoid mechanics, playable, realistic animals remain a rarity in common digital games, not to mention VR titles.

In our opinion, incorporating animals as player avatars into VR has the potential to unveil a set of novel game mechanics and maybe even lead to a ``beastly'' VR game genre. Furthermore, utilizing the abilities of animals such as flying as a bird or crawling as a spider could be significantly more engaging in VR due to the increased presence compared to non-VR games.

Apart from entertainment, we suggest that embodying animal avatars could help us to better understand the behavior of a certain creature, e.g., in an educational documentary. Such a strong bond to the creature caused by IVBO can also increase our involvement with environmental issues~\cite{ahn2016experiencing,berenguer2007effect} and our empathy for animals, which, in return, is transferrable to human-human empathy, as shown by Taylor et al.~\shortcite{taylor2005empathy}.

To capture the perspective of our society regarding these outlined applications, we administered an online survey with three sections: VR animals in general, animals in VR documentaries, and animals in VR games. Each section consisted of five questions about participants' experiences in that category and their general interest to give such a scenario a try. The questions were either yes/no, or on a 7-point Likert Scale ranging from 0 (\textit{totally disagree}) to 6 (\textit{totally agree}).

Thirty-seven subjects (21 female), aged 19 to 43 ($M = 26.43, SD = 5.67$), participated in the survey. Overall, 26 of the participants had prior experiences with VR, and 17 of them had already seen an animal in VR. However, only six subjects reported that they had had the chance to control a virtual animal. Our results indicate that the overall interest to try a VR application where animals play an important role is rather high ($M = 5.14, SD = 1.00$), and that subjects would like to observe, interact, and embody VR creatures (all $M > 4.50$).

The majority (33) had never seen a VR documentary about animals, but they said would like to try it ($M = 4.70, SD = 1.45$) and even embody a creature in such a documentary ($M = 4.24, SD = 1.80$). Participants also mostly agreed that embodying an animal might help them to better understand the animal's behavior ($M = 4.57, SD = 1.59$) and to increase their empathy toward that creature ($M = 4.89, SD = 1.45$).

Participants were also keen on trying a VR game with an animal avatar ($M = 4.81, SD = 1.45$). Playing in third-person perspective ($M = 3.65, SD = 1.86$) was preferred less than in first-person ($M = 4.38, SD = 1.66$). However, a paired-samples t-test shows that these differences are not significant.

Finally, the survey provided a multiple choice question as an opportunity for the subjects to tell us which animals they would like to experience in VR. The top three creature types were flying animals (birds, bats, etc.) with 32 votes, followed by typical mammals (lions, tigers, cats, dogs, etc.) with 30 votes, and by sea animals (dolphins, sharks, whales, etc.) with 25 votes. The fewest votes (7) were given to amphibians (toads, geckos, etc.). Combined with the results from our main IVBO study, we suppose that flying creatures indeed have the largest potential to fascinate users as embodiment targets in VR.



\subsection{Treatments of Animal Phobias in VR}
Confronted with an animal, empathy is not the only feeling that we might experience. On the negative side, we may feel fear and revulsion. Most of us know at least a few people with a fear of spiders, i.e., arachnophobia. Not only spiders, but also snakes, other insects, or even cats are also capable of inducing a specific phobia~\cite{shapse2008diagnostic}. 

When early studies indicated that the main component of therapy is a systematic exposure to the feared stimuli~\cite{marshall1977flooding}, researchers started to consider methods such as VR to establish \textit{in virtuo} exposures~\cite{carlin1997virtual,hoffman2003interfaces,garcia2002virtual,bouchard2006effectiveness,hoffman1998virtual} as an alternative for in vivo confrontation, e.g., stroking a real spider. Especially for children, such a VR approach can reduce the inhibition threshold~\cite{bouchard2014advances} in a therapy. For an in-depth overview related to VR treatment of such phobias, we point the reader to the book \textit{Advances in Virtual Reality and Anxiety Disorders} by Bouchard et al.~\shortcite{bouchard2014advances}.

We are curious whether embodiment might further reduce the inhibition threshold for an \textit{in virtuo} exposure. Therefore, we administered an online survey to gather a first impression regarding subjects' interest in and anxiety about undertaking a common VR treatment, i.e., being in a VR room with spiders, and an embodied VR treatment, i.e, being the spider itself. The statements were presented on a 7-point Likert Scale ranging from 0 (\textit{totally disagree}) to 6 (\textit{totally agree}).

Thirty-three participants responded to an email soliciting people with a fear of spiders. As a sanity check, we applied a simplified version of the Fear of Spiders Questionnaire~\cite{szymanski1995fear}. After such filtering, we kept 21 participants (14 female), aged 20 to 41 ($M = 27.14, SD = 4.90$), with half of the subjects (11) having prior VR experience.

Regarding the subjects' interest in trying such a treatment, embodied VR ($M = 4.24, SD = 1.73$) achieved higher scores compared to the common VR approach ($M = 3.19, SD = 1.72$). A paired samples t-test showed that the difference is significant; \textit{t}~(20)~=~-2.62, \textit{p}~=~.016. When asked about their anxiety for such treatments, the embodiment method ($M = 2.90, SD = 2.00$) also performed better (i.e., lower scores) than the common VR approach ($M = 3.52, SD = 1.72$). However, the difference is not significant according to a paired-samples t-test. In other words, our results indicate that including VR embodiment into the treatment process might have a positive impact on willingness to undertake such an exposure. Hence, in a next step, we propose to evaluate whether such a setup is also suitable as an \textit{in virtuo} exposure method, i.e., as a viable therapy.

To conclude, both studies support our assumption that animals in VR are underrepresented, yet bear potential for a plethora of CG and HCI application areas. As we want to shed more light on virtual creatures, our paper focuses on the IVBO aspect regarding virtual animals and explores benefits and drawbacks of various control approaches.


\section{Related Work on IVBO}

Let us first consider the most important terms related to VR environments. Acting in such environments adds \textit{immersion}~\cite{cairns2014immersion} as an influential factor to our experiences. Researchers~\cite{Biocca:1995:IVR:207922.207926,sherman2002understanding}, usually refer to immersion as the technical quality of VR equipment and use the term \textit{presence}~\cite{slater1995taking, slater2003note} when discussing the influence of such equipment on our perception. Lombard et al.~\shortcite{lombard1997heart} and IJsselsteijn et al.~\shortcite{IJsselsteijn} proposed several ways to measure presence. Apart from presence, i.e., the feeling of being there~\cite{heeter1992being}, immersive setups are capable of inducing the illusion of virtual body ownership (IVBO), also referred to as body transfer illusion, agency, or embodiment.


The illusion of virtual body ownership (IVBO)~\cite{lugrin2015anthropomorphism} is an adaption of the effect of body ownership (BO), a term coined by Botvinick et al.~\shortcite{botvinick1998rubber}. The authors conducted an experiment to induce the so-called rubber hand illusion, in which they hid the participant’s real arm and replaced it with an artificial rubber limb. Both arms were then simultaneously stroked by a brush, which produced the illusion of owning the artificial arm. This effect has gained great publicity and was further researched by Tsakiris et al.~\cite{tsakiris2005rubber}. These results eventually led to the first neurocognitive model regarding body ownership~\cite{tsakiris2010my}, which emphasized the interplay between external sensory stimuli and the internal model of our own body. Additional studies extended
these finding to other limbs and whole-body representations~\cite{ehrsson2007experimental,petkova2008if,lenggenhager2007video}.


The effect of BO was initially transferred to virtual environments for arms by Slater et al.~\shortcite{slater2008towards} and entire bodies by Banakou et al.~\shortcite{banakou2013illusory}. However, these early studies used the original visuotactile stimulation introduced by Botvinick et al.~\shortcite{botvinick1998rubber}. Later research introduced sensorimotor cues, i.e., the tracking of hand and finger movement~\cite{sanchez2010virtual}, which was reported to be more important than visuotactile cues~\cite{slater2010first}. This finding is essential for VR setups as it releases possible experiments from the need for tactile stimulations. Furthermore, these two types of different cues are completed by the so-called visuoproprioceptive cues. These cues are a series of different body representations and include subdimensions such as perspective, body continuity, posture and alignment, appearance, and realism. These different subdimensions are listed in the correct order of influence on the effect of IVBO~\cite{slater2009inducing,slater2010first,perez2012my,maselli2013building}, and together are sufficient for inducing the illusion of body ownership~\cite{maselli2013building}. Moreover, Maselli et al.~\shortcite{maselli2013building} reported the necessity of a first-person perspective. In sum, IVBO is induced by correct visuoproprioceptive cues. Misalignments and visual errors can be compensated for through the weaker aspects of sensorimotor and visuotactile cues. However, this effect can be observed with anthropomorphic characters as well as realistic representations~\cite{lugrin2015anthropomorphism, lin2016need,jo2017impact}.


Riva et al.~\shortcite{riva2014interacting} illustrated the current interest in significantly altering the morphology of our virtual representation by the following question: \textit{But what if, instead of simply extending our morphology, a person could become something else- a bat perhaps or an animal so far removed from the human that it does not even have the same kind of skeleton— an invertebrate, like a lobster?} Interestingly, embodying a bat is even being discussed in philosophy~\cite{nagel1974like}, which is one of the reasons we included that representation in our studies. Especially for animals that have few characteristics in common with our human body, the approach of sensory substitution.~\cite{bach2003sensory} is also a promising direction for IVBO research. For instance, we could replace the echolocation feature of a bat by visual or even tactile feedback in VR.

Recently, researchers have studied adapting and augmenting human bodies in VR. Kilteni et al.~\shortcite{kilteni2012extending} stretched the virtual arm up to four times its original length and were still able to confirm IVBO. Similarly, Normand et al.~\shortcite{normand2011multisensory} used IVBO to induce the feeling of owning a larger belly than in reality. These findings are in line with the work of Blom et al.~\shortcite{blom2014effects}, who reported that a strong spatial coincidence of real and virtual body part is not necessary for the illusion. Furthermore, researchers have determined that additional body parts are not necessarily destroying IVBO. Instead, it is possible to add a third arm and induce a double-touch feeling~\cite{ehrsson2009many,guterstam2011illusion}. 


Apart from additional arms, other body parts have also been added successfully: Steptoe et al.~\shortcite{steptoe2013human} reported effects of IVBO upon attaching a virtual tail-like body extension to the user’s virtual character. The authors further discovered higher degrees of body ownership when synchronizing the tail movement with the real body. Another prominent example of body modification that could be relevant for embodying flying animals is virtual wings. In that area, Egeberg et al.~\shortcite{Egeberg:2016:EHB:2927929.2927940} proposed several ways wing control could be coupled with sensory feedback, and Sikstr\"om et al.~\shortcite{sikstrom2014role} assessed the influence of sound on IVBO in such scenarios. Won et al.~\shortcite{won2015homuncular} further analyzed our ability to inhabit nonhumanoid avatars with additional body parts. Regarding realistic avatars, Waltemate et al.~\shortcite{waltemate2018impact} showed that customizable representations lead to significantly higher IVBO effects.


Strong effects of body ownership can produce multiple changes in the feeling or behavior of the user~\shortcite{jun2018full}, resembling the Proteus Effect by Yee et al.~\shortcite{yee2007proteus}. For instance, Peck et al.~\shortcite{peck2013putting} reported a significant reduction in racial bias when playing a black character. Additionally, virtual race can also affect the drumming style when playing virtual drums~\cite{kilteni2013drumming}. Other reactions are more childish feelings arising from child bodies~\cite{banakou2013illusory} or greater perceived stability due to a robotic self~\cite{lugrin2016avatar}. These findings demonstrate that IVBO is not just a one-way street but can be used to evoke specific feelings and attributes and possibly also change self-perception.

%One of the pioneering works regarding IVBO was conducted by Botvinick et al.~\cite{botvinick1998rubber}. The authors detected the so-called rubber hand illusion in an experiment where they hid the real arm of the participant behind a screen that displayed an artificial rubber limb. Both the real and virtual arms were then simultaneously stroked by a brush and participants reported to sense the touch on the virtual rubber hand. Further studies by Tsakiris et al.~\cite{tsakiris2005rubber,tsakiris2010my} revisited the rubber hand illusion and established a first neurocognitive model that described (virtual) body ownership as an interplay between multisensory input and internal models of the body. As a follow-up to virtual arms, researchers also extended the experiments to a whole body representation~\cite{ehrsson2007experimental,petkova2008if,lenggenhager2007video}. These early findings were followed by a number of insights regarding the visuotactile correlation~\cite{slater2008towards,sanchez2010virtual} and the involved visuoproprioceptive cues~\cite{slater2009inducing,perez2012my,maselli2013building,slater2010first} that help us to enhance the IVBO illusion. We will not go into more detail at that point, as the mentioned works focus on anthropomorphic representations and, thus, we cannot assume that the same principles are also applicable for animals.

%The interest to significantly alter the morphology of our virtual representation was recently sublimed by Riva et al.~\cite{riva2014interacting} in the following question: \textit{But what if, instead of simply extending our morphology, a person could become something else- a bat perhaps or an animal so far removed from the human that it does not even have the same kind of skeleton— an invertebrate, like a lobster?} Interestingly, embodying a bat is even something being discussed in philosophy~\cite{nagel1974like}, which is one of the reasons why we included that representation in particular into our studies. Especially for animals that have few in common with our human body, the approach of sensory substitution.~\cite{bach2003sensory} is also a promising direction for IVBO research. For instance, we could replace the echolocation feature of a bat by visual or even tactile feedback in VR.
%
%Targeting less exotic representations, researchers have determined that adding virtual body parts does not necessarily destroy the IVBO experience. For instance, Guterstam et al.~\cite{guterstam2011illusion} experimented with a third arm that evoked a duplicate touch feeling, while Normand et al.~\cite{normand2011multisensory} rather focused on creating the illusion of having an enlarged belly. Won et al.~\cite{won2015homuncular} further analyzed our ability to inhabit non-humanoid avatars with additional body parts.

%Steptoe et al.~\cite{steptoe2013human} added a tail-like, controllable body part to the avatar representation. The authors concluded that the tail movements need to be synchronized to provide a higher degree of body ownership. Another prominent example of body modification that could be relevant for embodying flying animals are virtual wings. In that area, Egeberg et al.~\cite{Egeberg:2016:EHB:2927929.2927940} exposed several ways how wing control could be coupled with sensory feedback, and Sikstr\"om et al.~\cite{sikstrom2014role} assessed the influence of sound on IVBO in such scenarios. Regarding realistic avatars, Waltemate et al.~\cite{waltemate2018impact} showed that customizable representations lead to significantly higher IVBO effects.

We point readers to the recent work in progress by Roth et al.~\shortcite{roth2017alpha} regarding IVBO experience. In particular, the paper presented a IVBO questionnaire based on a fake mirror scenario study. The authors suggested acceptance, control, and change as the three factors that determine IVBO. In our experiments, we administered the proposed questionnaire as we were curious to see how it performs for animal avatars. Our research follows up on the works-in-progress paper by Krekhov et al.~\shortcite{krekhov2018anim}. The authors conducted a preliminary, explorative study with eight participants, and, by applying the alpha IVBO questionnaire~\cite{roth2017alpha}, concluded that IVBO might indeed work for animal avatars. We significantly extend that apparatus to gather more insights and to produce reliable results, and also to introduce two additional surveys about virtual animals to explore the overall benefits of such research.

A body of literature related to the control of animal avatars should be mentioned in this context. Leite et al.~\shortcite{leite2012shape} experimented with virtual silhouettes of animals that were used like shadow puppets and controlled by body motion. For 3D cases, Rhodin et al.~\shortcite{rhodin2014interactive} applied sparse correspondence methods to create a mapping between player movements and animal behavior and tested their approach with species such as spiders and horses. As a next step, Rhodin at al.~\shortcite{rhodin2015generalizing} experimented with the generalization of wave gestures to create control possibilities for, e.g., caterpillar crawling movements. Our research extends these methods by presenting additional mechanisms tailored to animal avatar control.



\begin{figure}
\centering
\includegraphics[width=1.0\columnwidth]{figures/modes}
\caption{Two of our avatars in first-person (top) and third-person (bottom) modes in front of a wall-sized mirror.}
\label{fig:modes}
\end{figure}




%Our work can be classified as research in the areas of IVBO and player experience in VR. We assume that the latter concepts are familiar to our community, hence we only briefly address them before going into detail regarding IVBO. Player experience~\cite{Wiemeyer.2016} refers to aspects such as challenge, competence, immersion, and flow. The process of measuring such components was described by, e.g., IJsselsteijn et al.~\cite{ijsselsteijn2007characterising, ijsselsteijn2008measuring} and sublimed into the Game Experience Questionnaire~\cite{IJsselsteijn.2013}. 
%
%Playing in a VR setup adds \textit{immersion}~\cite{cairns2014immersion} as an influential factor. As often done by researchers (cf. \cite{Biocca:1995:IVR:207922.207926, sherman2002understanding}), we refer to immersion as the technical quality of a VR equipment and use the term \textit{presence}~\cite{slater1995taking, slater2003note} when talking about the influence of such equipment on our perception. Several ways to measure presence were exposed by, e.g., Lombard et al.~\cite{lombard1997heart} and IJsselsteijn et al.~\cite{IJsselsteijn}. Apart from presence, i.e., the feeling of being there~\cite{heeter1992being}, immersive setups are capable of inducing the illusion of virtual body ownership, also referred to as body transfer illusion, agency, or embodiment.


\begin{table*}
  \caption{Evaluated control modes for virtual animals.}
  \label{tab:controls}
  \begin{tabular}{l c p{8.7cm}}
    \toprule
    Mode & Evaluated avatars & Description\\
    \midrule
    first-person perspectives: & & \\
    \ \ \ \ \ \ full body \textbf{(FB)} & human, bat, spider, tiger & User's posture is mapped to the whole virtual body. Mapping depends on the animal; see \FG{fig:mapping} and \FG{fig:teaser} for examples.\\[0.15cm]
    \ \ \ \ \ \ half body \textbf{(HB)} & spider, tiger & User's legs mapped to all limbs of an animal. \\[0.15cm]
    third-person perspectives: & & \\
    \ \ \ \ \ \ user centered \textbf{(3CAM)} & spider & The animated avatar is locked into a position in front of the user. \\[0.15cm]
    \ \ \ \ \ \ agent controlled \textbf{(3NAV)} & spider & The avatar is an autonomous agent following a target in front of the user. \\[0.15cm]
    \ \ \ \ \ \ avatar centered \textbf{(3FOL)}  & spider & The user is rotated around the animated avatar when turning. \\
    \bottomrule
  \end{tabular}
\end{table*}



\section{Animal Embodiment}


As we can see from related work, body ownership requires as much sensory feedback as possible. Hence, if we want to evoke such experiences in a VR application, a simple gamepad control is probably not adequate. Instead, prior research has shown that either proprioceptive cues or sensorimotor cues are necessary to induce proper levels of VBO. However, providing such cues is challenging for nonhuman characters as usually no straightforward control mapping exists between the participant and the virtual creature. 

In contrast to humans, animals come in various shapes, postures, and types, which makes it difficult to design a universal solution for avatar control. Therefore, our experiment includes multiple models combined with different types of control to gather diverse insights into animal embodiment.

Animal and human bodies differ in three main subdomains that are critical for successful body ownership: skeleton, posture, and shape, as can be seen in \FG{fig:mapping}. Certain animals, such as bats, share a human posture and skeleton but use scaled arms or legs and therefore vary in the natural shape, i.e., differ in terms of proportions. Other creatures such as tigers or dogs have an almost human skeleton, including the same number of limbs. However, they differ in the natural posture by walking on all fours. Finally, other species show a completely different skeleton and differ in the limb count. An appropriate example is a spider, which has eight legs attached to its head segment. To cover these different degrees of anthropomorphism, we have chosen tigers, spiders, and bats as our testbed species. In addition, we added a human avatar to compare our results with humanoid IVBO scores.



%As we can see from related work, body ownership requires as much sensory feedback as possible. Hence, if we want to evoke such experiences in a VR game, a simple gamepad control is probably not adequate. Another well-established mechanism is to map player posture to the virtual representation. This is a challenging task, as most animals are not bipedal and there is no straightforward posture mapping. 
%
%We focused on three example animals (tiger, bat, spider) to cover a broad range of possible avatars as shown in \FG{fig:animals}. Tigers expose a similar skeleton and similar limb proportions compared to humanoids. Bats have a similar skeleton, but very different proportions. Finally, spiders have a completely different skeleton and an increased number of limbs with non-human proportions.



\subsection{Mapping Approaches}

We designed and evaluated multiple control modes and mapping approaches, as summarized in Table~\ref{tab:controls}. Even though prior work, e.g., by Debarba et al.~\shortcite{galvan2015characterizing}, underpins the superiority of first-person mappings regarding IVBO, that finding has not yet been confirmed for nonhuman embodiment. Hence, we decided to use both first-person and third-person perspectives (cf. \FG{fig:modes}) in our experiment to contribute to the perspective discussion.

The third-person perspective provides the advantage that subjects see their avatars standing right in front of them. However, that perspective is challenging when subjects rotate around themselves. For instance, in current non-VR games, the camera---or, in our case, the player---slides around the avatar to maintain the over-the-shoulder viewport.  This approach has been tested as one possible mode and named 3FOL. Another option is to use the subject as the rotational center and turn the animal around (3CAM). This mode is proposed to induce less cybersickness~\cite{laviola2000discussion} but lacks realism as the avatar slides sideways around the subject. Finally, this approach can be changed to enhance the visual quality by implementing a loose coupling: the animal avatar would be controlled by an agent trying to stay in front of the subject. This concept has the advantage that movement and rotation are chosen optimally to look natural while preserving the rotational center of 3CAM. We refer to this approach as 3NAV. In contrast to these three third-person perspectives, the first-person perspective is not affected by different rotational centers because the subject and the avatar share the same position.


Apart from different perspectives, our approaches also differ in the type of mapping that is applied. The tiger is usually walking on all fours. Hence, a subject imitating and becoming the animal could move the same with all four limbs being mapped to the tiger body. However, this full-body (FB) tracking is assumed to be somewhat exhausting as it forces participants to crouch on the floor. As an alternative, we introduce half-body (HB) tracking : the subjects stand or walk in an upright position, watch through the eyes of their animal, and have their lower body parts mapped to all of the animal’s limbs. For instance, in case of a tiger, one human leg corresponds to two of the animal's pawns. This variation preserves the amount of sensory feedback while reducing the necessary physical effort. Another approach---the one we used for the third-person perspectives---is to avoid posture tracking and replace it with predefined avatar animations, only keeping the subject's position and orientation in sync.

% ``parentheses''




\begin{figure*}[t!]
\centering
\includegraphics[width=2.1\columnwidth]{figures/mapping}
\caption{Three virtual animals and the human avatar (right) that was used as the reference for IVBO comparisons. The animals were chosen such that they differ from humanoids in IVBO-critical domains, i.e., shape (bat), skeleton (spider), and posture (tiger, spider). The subjects in the image were captured in FB mode, i.e., they had to crouch to control the tiger and the spider.}
\label{fig:mapping}
\end{figure*}


\subsection{Testbed Scenario}

We utilized a combination of Unity3D~\shortcite{unity} and HTC Vive~\shortcite{vive}, including additional Vive trackers positioned at the hip and both ankles to enable full-body positional tracking. The HB and FB modes required custom avatar poses depending on tracker positions and rotations. Therefore, we experimented with different approaches based on inverse kinematics (IK)~\cite{buss2004introduction}. Physical models typically used for ragdoll systems and iterative solvers tended to jitter and flicker upon combining them with the VR tracking. As these issues were partially caused by unavoidable tracking errors, these approaches did not suit the situation. Instead, we applied a combination of closed-form and iterative solvers to achieve more stability at the cost of limited rotational movement. 

As depicted in \FG{fig:scene}, we placed our experiment in a stereotypical zoo where the participants were locked inside an arena-like cage filled with different interactive items such as cans, crates, or tires. Moreover, we installed a virtual wall-sized mirror to enhance the VBO illusion~\cite{latoschik2016fakemi}. So we relied on the same testbed scene for all conditions, animals were scaled to roughly equal, human-like dimensions.




\subsection{Hypotheses and Research Questions}

Our main goal is to explore animal embodiment by evaluating the five proposed mapping approaches with different animals. We want to see how potential users perceive the different control modes and what they like or dislike about our animal avatars in VR. Furthermore, we hypothesize that, similar to humanoid IVBO findings~\cite{galvan2015characterizing}, third-person modes for animals are inferior to the first-person perspective. To summarize, our questions and hypotheses are the following:
\begin{itemize}
  \setlength{\itemsep}{2pt}
  \setlength{\parskip}{0pt}
  \setlength{\parsep}{0pt}
%TODO Als Aufzählung formatieren
\item RQ1: How do first-person modes (FB and HB) for animals perform regarding IVBO compared to a human avatar?
\item RQ2: Do our creature types differ regarding IVBO and user valuation?
\item RQ3: Is there any difference between FB and HB for the same animal?
\item H1: First-person modes (FB and HB) significantly outperform third-person modes (3CAM, 3NAV, 3FOL) regarding induced IVBO.
\end{itemize}




\subsection{Procedure and Applied Measures}


We conducted a within-subjects study in our VR lab and tested the following conditions: FB human (as reference for IVBO), FB spider, FB tiger, FB bat, HB spider, HB tiger, 3CAM spider, 3NAV spider, and 3FOL spider. We excluded the HB bat case because that animal can be controlled in an upright pose in FB. Thus, we do not see any advantage to using the lower body only. We limited the third-person modes to one animal because these approaches behave the same for all animals.




Upon the participants' arrival, we administered a general questionnaire assessing age, gender, digital gaming behavior, and prior experiences with VR systems. For each condition, we told the participants to move around in the virtual arena and experiment with their virtual representation. For instance, subjects were able to move and drag various objects, such as crates, pylons, and tires. Subjects stayed in the virtual world for around five minutes for each condition. This duration is a typical choice for IVBO studies~\cite{tsakiris2005rubber} despite the finding that even 15 seconds may be enough to induce body ownership~\cite{lloyd2007spatial}. 

We decided against performing threat tests for capturing IVBO, as the sequence effects in our case would be too significant. Note there is no unified procedure for measuring IVBO and a threat test is not the only possibility~\cite{kilteni2012sense,roth2017alpha}. Instead, we decided to use the alpha IVBO questionnaire by Roth et al.~\shortcite{roth2017alpha}, and also checked its reliability by calculating Cronbach’s alpha for all subscales (all alphas > 0.81).

\begin{figure}[b]
\centering
\includegraphics[width=1.0\columnwidth]{figures/scene}
\caption{We chose a virtual zoo for our testbed scenario. The cage is equipped with a wall-sized mirror to enhance IVBO.}
\label{fig:scene}
\end{figure}


\begin{figure*}[t!]
\centering
\includegraphics[width=2.1\columnwidth]{figures/diagram}
\caption{Mean scores and standard deviations for the three IVBO dimensions: acceptance, control, and change.}
\label{fig:diagram}
\end{figure*}

We administered the alpha IVBO questionnaire after each condition. Answers were captured on a 7-point Likert Scale ranging from 0 (\textit{totally disagree}) to 6 (\textit{totally agree}). In particular, the questionnaire captures the three dimensions acceptance, control, and change. Acceptance reflects self-attribution and owning of the virtual body by statements such as: \textit{I felt as if the body parts I looked upon were my body parts}. Control mostly focuses on the correct feedback and agency. One example is: \textit{I felt as if I was causing the movement I saw in the virtual mirror}. Finally, change measures self-perception and is usually triggered when the avatar differs much from the user. Three subitems focus on changes during the experiment (e.g., \textit{At a time during the experiment I felt as if my real body changed in its shape, and/or texture}), whereas another three subitems capture after-effects (e.g., \textit{I felt an after-effect as if my body had become taller/smaller}).





% by the following questions:
%
%\textbf{Acceptance:} \textit{I felt as if the body I saw in the virtual mirror might be my body} [myBody], \textit{I felt as if the body parts I looked upon where my body parts} [myBodyParts], \textit{The virtual body I saw was humanlike/animallike} [hum,anness].
%
%\textbf{Control:} \textit{The movements I saw in the virtual mirror make seemed to be my own movements} [myMovement], \textit{I enjoyed controlling the virtual body I saw in the virtual mirror} [bodyControlEnjoyment], \textit{I felt as if I was controlling the movement I saw in the virtual mirror} [controlMovements], \textit{I felt as if I was causing the movement I saw in the virtual mirror} [causeMovements].
%
%\textbf{Change:} \textit{} [ownOtherbody], \textit{} [myBodyChange], \textit{} [myBodyCheck], \textit{} [echoHeavyLight], \textit{} [echo-Tall-Small], \textit{} [echo-LargeThin], \textit{} []
%
%\begin{itemize}
%\item \textit{I felt as if the body I saw in the virtual mirror might be my body} [myBody].
%\item \textit{I felt as if the body parts I looked upon where my body parts} [myBodyParts].
%\item \textit{The virtual body I saw was humanlike/animallike} [humanness].
%\end{itemize}


We extended the questionnaire by a number of additional custom questions and statements to capture fascination (\textit{The overall experience was fascinating}), ease of control (\textit{I coped with the control of the avatar}), and fatigue (\textit{Controlling the avatar was exhausting}). We used the same scales as for the IVBO questions. Furthermore, we conducted semistructured interviews after each control mode, i.e., after FB, HB, 3CAM, 3NAV, and 3FOL. In particular, we asked subjects what they liked best/least and why, whether they could imagine such controls in a VR game, and how we could further enhance that mode. Upon completion of all conditions, participants had the chance to provide general feedback regarding animal avatars and tell us their favorite animals to be included in the next experiments.





\subsection{Results}

Twenty-six subjects (13 female) with a mean age of $23.46$ ($SD=7.06$) participated in our study. Most participants (21) reported playing digital games at least a few times a month, and the majority (21) had used VR gaming systems. 

To address our hypothesis and research questions, we compare all nine conditions regarding their IVBO performance for acceptance, control, and change. All investigated parameters were approximately normally distributed according to Kolmogorov-Smirnov tests. Hence, we used one-way repeated measures ANOVA to compare the measured IVBO  values outlined in \FG{fig:diagram}. The outcomes differed significantly in all three dimensions, i.e., acceptance, \textit{F}~(4.86, 122.14)~=~18.23, \textit{p}~<~.001, control, \textit{F}~(3.64, 90.95)~=~18.54, \textit{p}~<~.001, and change, \textit{F}~(4.11, 102.73)~=~14.54, \textit{p}~<~.001. Post hoc Bonferroni tests provided additional insights into these differences:

\textbf{Acceptance:} the human avatar ($M = 2.79, SD = 1.31$) was rated significantly lower than FB bat ($M = 4.33, SD = 1.10$) and HB spider ($M = 3.63, SD = 1.29$) with $p < .01$ in both cases. 

\textbf{Control:} FB bat ($M = 5.11, SD = 0.82$) achieved significantly higher scores than all other modes, whereas 3NAV ($M = 2.41, SD = 1.38$) and 3FOL ($M = 2.55, SD = 1.22$) performed significantly worse than each first-person mode (all $p < .05$). 

\textbf{Change:} All modes had rather low values, as can be seen in \FG{fig:diagram}. FB bat ($M = 2.22, SD = 1.28$) was rated significantly better (all $p < .01$) than HB tiger ($M = 1.57, SD = 1.18$), 3CAM ($M = 1.02, SD = 1.34$), 3NAV ($M = 0.79, SD = 1.10$), and 3FOL ($M = 0.92, SD = 0.90$). Also, all first-person modes significantly outperformed 3NAV and 3FOL (all $p < .05$).

To create a better picture for RQ2, we also considered our custom questions and statements summarized in \FG{fig:diagram2}. Our conditions significantly differed regarding fascination, \textit{F}~(3.68, 92.08)~=~2.99, \textit{p}~=~.026, ease of control, \textit{F}~(5.10, 127.58)~=~8.58, \textit{p}~<~.001, and fatigue, \textit{F}~(5.43, 135.76)~=~13.46, \textit{p}~<~.001.  Post hoc Bonferroni tests revealed the following details:


\textbf{Fascination:} FB bat ($M = 5.27, SD = 0.78$) significantly outperformed (all $p < .05$) FB human ($M = 4.42, SD = 1.24$), HB tiger ($M = 4.08, SD = 1.55$), 3CAM ($M = 4.27, SD = 1.51$) and 3FOL ($M = 3.88, SD = 1.80$). 

\textbf{Ease of Control:} Again, FB bat ($M = 5.38, SD = 0.75$) was perceived very positively and had significantly higher scores (all $p < .01$) than all other modes except 3CAM ($M = 4.65, SD = 1.47$), which was ranked second. In contrast, 3FOL ($M = 3.04, SD = 1.71$) produced most control difficulties, which resulted in significantly lower scores (all $p < .05$) than FB human, FB bat, HB spider, and 3CAM. 

\textbf{Fatigue:} Similarly, 3FOL ($M = 3.65, SD = 2.00$) was most exhausting and performed significantly worse (all $p < .05$) than all modes except FB tiger ($M = 3.58, SD = 1.39$) and FB spider ($M = 3.35, SD = 1.55$). The two latter modes were also rated significantly inferior (all $p < .05$) to the remaining modes, which all stayed below 2 as mean value.



\begin{figure*}[t!]
\centering
\includegraphics[width=2.1\columnwidth]{figures/diagram2}
\caption{Mean scores and standard deviations for fascination (\textit{The overall experience was fascinating}), ease of control (\textit{I coped with the control of the avatar}), and fatigue (\textit{Controlling the avatar was exhausting}).}
\label{fig:diagram2}
\end{figure*}

%r3: fb bat vs hb bat, fb tiger vs hb tiger.
%h1: FB spider, HB spider, 3rd person, but also mention other fps.
%compare human avatar to results of FB spider, FB tiger, FB bat, HB spider, and HB tiger modes.



\subsection{Discussion}

% ``parentheses''

%\begin{itemize}
%  \setlength{\itemsep}{2pt}
%  \setlength{\parskip}{0pt}
%  \setlength{\parsep}{0pt}
%%TODO Als Aufzählung formatieren
%\item RQ1: How do first person modes (FB and HB) for animals perform regarding IVBO compared to a human avatar?
%\item RQ2: Do our creature types differ regarding IVBO and user valuation?
%\item RQ3: Is there any difference between FB and HB for the same animal?
%\item H1: First person modes (FB and HB) significantly outperform third person modes (3CAM, 3NAV, 3FOL) regarding induced IVBO.
%\end{itemize}


\subsubsection{\textbf{RQ1:} How do first-person modes (FB and HB) for animals perform regarding IVBO compared to a human avatar?}
In all three IVBO dimensions, ANOVA did not reveal any significant advantages of FB human over the animal first-person modes. On the contrary, for acceptance and control, the humanoid representation was significantly outperformed by FB bat, and, for acceptance, also by HB spider. Hence, our main observation is that IVBO should be applicable for nonhumanoid avatars that differ in shape, skeleton, or posture from our human body. 

However, we are aware that the appearance of the human avatar has also a strong impact on IVBO. For instance, customizing that representation~\cite{waltemate2018impact} could produce significantly higher IVBO scores for that condition. Thus, we do not want to exaggerate the generality of our finding, and rather state that animal IVBO has the potential to keep up with humanoid IVBO and, thus, is worth further, more detailed investigations.


\subsubsection{\textbf{RQ2:} Do our creature types differ regarding IVBO and user valuation?}

In our case, the clear ``winner'' regarding IVBO scores and our custom questions is the bat, a creature type that mostly differs in shape but maintains similar posture and skeleton compared to our body. This finding might be an indication that animals with human-like, upright postures are more suited for IVBO effects. The quantitative results also align with the interview feedback that we got during the evaluation: \textit{``The bat behaved exactly how I expected and it was intriguing to precisely control my wing movements because it appeared realistic to me''(\textbf{P7})}. Subjects often expressed their desire to utilize the flying capability: \textit{``I could feel more like a giant bat if I could fly by moving my arms and maybe lean forward to accelerate''(\textbf{P4})}.



Another finding is the different perceptions of FB spider and FB tiger modes. Participants reported that the tiger felt less engaging, and we recorded several similar statements such as the following: \textit{``The forepaws were too short, they even felt shorter than my real arms and I could not do much with them''(\textbf{P2})}. Of course, tiger paws are not shorter, but the tiger head position leads to such a distorted perspective. Hence, we suppose that perceiving virtual limbs as shorter than our real limbs feels rather limiting, whereas longer virtual limbs are classified as useful tools that enhance our interaction space in VR. This finding would also explain the supremacy of FB bat mode with wings as extended tools: \textit{``The long arms of the bat felt a bit like two long sticks I could use to reach more items''(\textbf{P18})}. 


\subsubsection{\textbf{RQ3:} Is there any difference between FB and HB for the same animal?} Our experiment did not reveal any significant differences in that regard, which is a surprising observation because HB reflects only half of our posture changes. HB spider overall achieved positive ratings that were close to FB bat and even significantly outperformed FB human regarding acceptance.

\begin{figure}[b]
\centering
\includegraphics[width=1.0\columnwidth]{figures/views}
\caption{The biggest difference of 3NAV (left) to 3CAM and 3FOL (right) occurs when subjects walk backwards because the avatar is suddenly facing and ``chasing'' them.
}
\label{fig:chase}
\end{figure}
Our custom question related to fatigue revealed that subjects perceived FB modes for spider and tiger control to be significantly more exhausting than their HB counterparts because they had to kneel and crouch on a yoga mat. For such types of animals, HB modes seem to be more promising because they expose the same amount of IVBO without being aggravating. However, one disadvantage of HB is the less direct mapping, i.e., subjects \textit{``felt limited regarding possible interactions as it is difficult to forecast the avatar behavior sometimes''(\textbf{P2})}. Regarding missing control in HB, two participants mentioned that, in case of a spider, they would like \textit{``to control each limb separately, maybe even with finger movements''(\textbf{P5})}.


\subsubsection{\textbf{H1:} First-person modes (FB and HB) significantly outperform third-person modes (3CAM, 3NAV, 3FOL) regarding induced IVBO} Overall, all third-person modes achieved rather low scores in all IVBO dimensions. In particular, for control and change, the 3NAV and 3FOL modes were significantly outperformed by all first-person perspectives, which mostly supports our hypothesis and is in line with humanoid research~\cite{galvan2015characterizing, maselli2013building}. Hence, if a higher IVBO is desired, controlling an animal in first-person mode is advantageous.

The scores for 3CAM were not significantly lower compared to first-person modes, which renders that approach a viable alternative if first-person is not possible. Also, in the interview, most subjects reported slightly preferring the 3CAM mode over 3NAV and 3FOL. In the 3NAV condition, subjects perceived the avatar to be \textit{``controlled telepathically''(\textbf{P8})} and to \textit{``orbit the player''(\textbf{P2})}. One participant was slightly surprised at one point (cf. \FG{fig:chase}): \textit{``When I walked backward, the spider suddenly looked at me and seemed to chase me''(\textbf{P1})}. We suggest that 3CAM and 3NAV are both suited for VR applications, yet 3NAV resembles more companion-like behavior rather than an avatar representation. In contrast, we do not recommend using the 3FOL mode, as it is capable of evoking dizziness, as was confirmed by two participants. Especially regarding the question about how exhausting the control was, 3FOL performed significantly worse compared to other third-person perspectives. Hence, even though it is widely used in non-VR games, we do not see any notable advantages of 3FOL in a VR setup.



%The results depicted in \FG{fig:diagram} show that FB and HB control modes can indeed create the illusion of owning an animal body in VR. Both modes did not perform worse than the reference virtual human. Note that there are no big differences regarding IVBO when comparing FB and HB, which was a surprising observation because HB reflects only half of our posture changes.
%
%Our custom question related to effort revealed that subjects perceived FB modes for spider and tiger control as rather exhaustive because they had to kneel and crouch on a yoga mat. For such type of animals, HB modes seem to be more promising, as they expose the same amount of IVBO without being aggravating. However, one disadvantage of HB is the less direct mapping, i.e., subjects \textit{``felt limited regarding possible interactions as it is difficult to forecast the avatar behavior sometimes''(\textbf{P2})}. Regarding missing control in HB, two participants mentioned that, in case of a spider, they would like \textit{``to control each limb separately, maybe even with finger movements''(\textbf{P5})}.
%
%Participants explained that the bat~(FB) was their favorite: \textit{``The bat behaved exactly how I expected and it was intriguing to precisely control my wing movements because it appeared realistic to me''(\textbf{P7})}. Subjects often expressed their desire to utilize the flying capability: \textit{``I could feel more like a giant bat if I could fly by moving my arms and maybe lean forward to accelerate''(\textbf{P4})}.
%
%\begin{figure}[b]
%\centering
%\includegraphics[width=1.0\columnwidth]{figures/views}
%\caption{The biggest difference of 3NAV (left) to 3CAM and 3FOL (right) occurs when subjects walk backwards because the avatar is suddenly facing and ``chasing'' them.
%}
%\label{fig:chase}
%\end{figure}
%
%Another finding to be mentioned is the different perception of FB spider and FB tiger modes. Participants reported that the tiger felt less engaging, and we recorded several similar statements such as the following: \textit{``the forepaws were too short, they even felt shorter than my real arms and I could not do much with them''(\textbf{P2})}. Of course, tiger paws are not shorter, but the tiger head position leads to such a distorted perspective. Hence, we suppose that perceiving virtual limbs as shorter than our real limbs feels rather limiting, whereas longer virtual limbs are rather classified as useful tools that enhance our interaction space in VR.
%
%In all three TPM, the measured IVBO effects were rather low compared to FPM. Most participants reported to slightly prefer the 3CAM mode over 3NAV. In the 3NAV condition, subjects perceived the avatar to be \textit{``controlled telepathically''(\textbf{P8})} and to \textit{``orbit the player''(\textbf{P2})}. One participant was slightly surprised at one point (similar to \FG{fig:chase}): \textit{``when I walked backward, the spider suddenly looked at me and seemed to chase me''(\textbf{P1})}. We suggest that 3CAM and 3NAV are both suited for games (cf. \FG{fig:d3}), yet 3NAV resembles more a companion-like behavior rather than an avatar representation. In contrast, we do not recommend the usage of 3FOL mode, as it is capable of evoking dizziness, as was confirmed by two participants. Especially regarding the question about how exhausting the control was, 3FOL performed notably worse compared to 3CAM and 3NAV.

%
%\TODO{maybe remove and discuss in online survey}
%We also recorded a wish list of animals that subjects would like to control: birds, elephants, bunnies, fishes, slugs, apes, dinosaurs, and sharks. Furthermore, one participant requested a mapping of facial expressions.
%
%
%\subsubsection{Children and Animal Avatars}
%
%Both children answered to most IVBO-related questions with 6 (\textit{totally agree}). This could be an indicator that children might be more affected by such VR phenomena and we suggest follow-up research in that direction. In contrast, third person modes were perceived as least exciting, even if one \textit{``could see lots of details of the animal''(\textbf{P10})}.

\subsubsection{Design Implications} The outcomes of our experiment allow the formulation of some early design considerations for further research. In first place, we argue that a 1:1 full-body mapping is not a key requirement for IVBO, as half-body approaches often achieved similarly high scores. This observation is especially important for the design of animals with significantly different skeletons that cannot be mapped to our human anthropology, as we can still induce the IVBO effect under such conditions. 

In general, we note that half-body approaches that map one of our legs to multiple animal limbs should be considered instead of forcing users in a non-upright position such as crouching. Half-body maximizes the IVBO effect compared to less direct mapping modes, yet removes the discomfort induced by full-body controls. However, using our legs only limits the interaction precision, and we recommend mapping fine-grained manipulation tasks to our hands, if required. For instance, we could map the two front limbs of the spider to our arms, which allows users to execute more precise actions such as picking up and holding objects.

Another observation is related to the choice of perspective. Although third-person approaches were inferior to first-person modes regarding IVBO, we argue that third-person offers a viable option for rapid prototyping of VR animal applications. First-person modes presume a precise motion mapping to perform well, which usually requires tuned IK solutions and knowledge of animal kinesiology. In contrast,  third-person controls---probably due to the weaker IVBO---can rely on simple, predefined avatar animations: in our experiment, participants have not noticed any difference nor reported any resentment related to the unsynchronized movements.





\section{Conclusion and Future Work}

Backed by our supplementary studies, we underpinned the large potential of animal avatars for VR research and applications, be it for education, entertainment, or as an add-on in treatment of animal phobias. To provide a starting point for future research, we proposed a number of different control modes for upright/flying species, four-legged mammals, and arthropods. Our evaluation revealed that IVBO can be considered for nonhumanoid avatars and led us to a first set of design implications in that area. 

In particular, we conclude that half-body tracking is a viable alternative to control animals that are not in an upright position as it offers a promising trade-off between fatigue and IVBO. For that reason, we suggest examining such half-body approaches in more depth. To provide higher degrees of control, a combination with sensory substitution~\cite{bach2003sensory} might be a viable approach for future research. Finally, as desired by the majority of participants, we propose to enhance the virtual representations with appropriate capabilities such as flying and see how this would impact IVBO. Hereby, the ultimate goal is the construction of a zoological IVBO framework that would support researchers and practitioners in designing meaningful virtual animals.



% DO NOT INCLUDE ACKNOWLEDGMENTS IN AN ANONYMOUS SUBMISSION TO SIGGRAPH 2019
%\begin{acks}
%
%The authors would like to thank Dr. Maura Turolla of Telecom
%Italia for providing specifications about the application scenario.
%
%The work is supported by the \grantsponsor{GS501100001809}{National
%  Natural Science Foundation of
%  China}{http://dx.doi.org/10.13039/501100001809} under Grant
%No.:~\grantnum{GS501100001809}{61273304\_a}
%and~\grantnum[http://www.nnsf.cn/youngscientists]{GS501100001809}{Young
%  Scientists' Support Program}.
%
%
%\end{acks}

% Bibliography
\bibliographystyle{ACM-Reference-Format}
\bibliography{vranim-bibliography}






\end{document}
